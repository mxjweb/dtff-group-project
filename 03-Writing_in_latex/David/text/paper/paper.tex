\documentclass[10pt,a4paper]{article}

\usepackage[
    backend=biber,
    style=bwl-FU,
    url=false,
    doi=false,
    eprint=false]{biblatex}
\addbibresource{bibliography.bib}

%sections on new pages
\usepackage{titlesec}
\newcommand{\sectionbreak}{\clearpage}

\usepackage{amsmath}
\usepackage{amssymb}
\usepackage{graphicx}
\usepackage{threeparttable}
\usepackage{blindtext}
\usepackage{booktabs}

\title{%
  Digital Tools for Finance \\
  \large What you should know!}
\author{David Jaggi\thanks{University of Zurich, Switzerland}\;\thanks{I am extremely grateful to Igor Pozdeev for his unwavering support and patient advice.}}
\date{December 2021}

\begin{document}
\maketitle
\begin{abstract}
    \blindtext
\end{abstract}
\newpage
\blindtext[3]
\textcite{gormsenCoronavirusImpactStock2020}
\begin{figure}[h!]%
    \centering
    \includegraphics[width=6cm]{text/paper/github_logo.png}%
    \caption{GitHub Logo}%
    \footnotesize Numbers represent search interest relative to the highest point on the chart for the given region and time. A value of 100 is the peak popularity for the term. A value of 50 means that the term is half as popular. A score of 0 means there was not enough data for this term.
    \label{fig:github_logo}%
\end{figure}
\blindtext[3]
\textcite{kozlowskiTailThatWags2020}
\begin{table}
    \centering  
  \begin{threeparttable}
    \caption{Sample ANOVA table}
     \begin{tabular}{lllll}
        \toprule
        Stubhead & \( df \) & \( f \) & \( \eta \) & \( p \) \\
        \midrule
                 &     \multicolumn{4}{c}{Spanning text}     \\
        Row 1    & 1        & 0.67    & 0.55       & 0.41    \\
        Row 2    & 2        & 0.02    & 0.01       & 0.39    \\
        Row 3    & 3        & 0.15    & 0.33       & 0.34    \\
        Row 4    & 4        & 1.00    & 0.76       & 0.54    \\
        \bottomrule
     \end{tabular}
    \begin{tablenotes}
      \small
      \item Additional information on the data and methods used.
    \end{tablenotes}
  \end{threeparttable}
\end{table}
\blindtext[4]
\printbibliography
\end{document}