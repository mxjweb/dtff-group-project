\documentclass[10pt,aspectratio=169]{beamer}

\usetheme{Berlin}
\usefonttheme{professionalfonts}
\usecolortheme{orchid}

\usepackage{blindtext}
\usepackage{amsmath}

\title{Digital Tools for Finance}
\subtitle{If a tree fell in your random forest, would anyone notice?}
\author{David Jaggi}
\institute{University of Zurich}
\date{December 2021}

\begin{document}
% ---------------------------------------------------------------------------
\frame{\titlepage}
% ---------------------------------------------------------------------------
\begin{frame}{Frame 1}
    \framesubtitle{Subtitle 1}
    \begin{figure}
    \centering
        \begin{minipage}{.45\textwidth}
          \centering
          \includegraphics[width=.75\linewidth]{example-image-a}
          \caption{Very nice image A.}
          \label{fig:test-a}
        \end{minipage}%
        \begin{minipage}{.45\textwidth}
          \centering
          \includegraphics[width=.75\linewidth]{example-image-b}
          \caption{Also nice image B.}
          \label{fig:test-b}
        \end{minipage}
    \end{figure}
\end{frame}
% ---------------------------------------------------------------------------
\begin{frame}{Frame 2}
    \framesubtitle{Subtitle 2}
    \begin{theorem}{Example Theorem}
        Let \(f\) be a function whose derivative exists in every point, then \(f\) 
        is a continuous function.
    \end{theorem}
\end{frame}
% ---------------------------------------------------------------------------
\begin{frame}{Frame 3}
    \framesubtitle{Subtitle 3}
        \begin{gather}
            1+1=2 \\
            2+2=4
        \end{gather}
\end{frame}
% ---------------------------------------------------------------------------

\end{document}